\documentclass[12pt]{article}

\usepackage[letterpaper, hmargin=0.75in, vmargin=0.75in]{geometry}
\usepackage{float}
\usepackage{listings}
\usepackage{verbatim}
\usepackage{graphicx}
\usepackage{url}
\usepackage[toc,page]{appendix}

\setlength{\parskip}{8pt}
\pagestyle{empty}

\title{TrueType Font Program Table Tree Shaking}

\author{
  Eduardo Hauck dos Santos\\
  \texttt{eduardohauck@gmail.com}
  \and
  Patrick Lam\\
  \texttt{p.lam@ece.uwaterloo.ca}
}

\lstset{frame=single}

\begin{document}

\maketitle

\section{Motivation}
TrueType is an outline font standard desgined by Apple and Microsoft as
a competitor to Adobe's Type 1 fonts. In TrueType fonts, each glyph is
described by a set of outlines. To ensure excellent font rendering at typical
screen resolutions, font designers often include font hints to adjust the display of
problematic outlines. TrueType font hints are expressed in a TrueType-specific
stack-based bytecode language.

Websites or documents with embedded fonts would benefit
from smaller font files, specially on mobile devices with network or
memory limitations. However, keeping
font size minimal is a challenge due to the lack of optimizing techniques
for its bytecode. 

TrueType fonts files in the wild often contain bytecode that is never
executed. This work investigates how much we can shrink TrueType font
files if we remove some of this bytecode---specifically, unnecessary
function definitions from the font program table.

\section{Description}

Our tree shaking tool reduces the font program table of a
TrueType font by removing function definitions that are never called
during execution. The output of the tree shake is a new font file, which will be
smaller if our tree shaking tool manages to remove uncalled functions.

The tree shake can be combined with subsetting techniques to achieve an
even better relative reduction with respect to the original size of the font file. Font
subsetting takes a font as input and produces a new font as output. The new font
contains only a subset of the original font's glyphs. It can be used
for documents whose selection of glyphs from a particular font remain static after
initial rendering. Subsetting may also remove the need for functions
used by specific glyphs.

To decide which functions can be eliminated, we need to identify the
functions defined and all the function calls made during execution.

Function definitions are located within the font program table in the
form of stack based code. Since this table contains exclusively function
definitions (i.e. no instructions other than function definition
instructions are executed) we can obtain and store function labels without
any analysis.

\begin{figure}[ht!]
  \begin{center}
  \begin{minipage}{.4\textwidth}
\begin{verbatim}
PUSH[ ]  /* 2 values pushed */
2 1
FDEF[ ]
PUSH[ ]
14
SWAP[ ]
WCVTP[ ]
ENDF[ ]
FDEF[ ]
DUP[ ]
RCVT[ ]
PUSH[ ]
0
RS[ ]
ADD[ ]
WCVTP[ ]
ENDF[ ]
\end{verbatim}
  \end{minipage}
  \end{center}
\caption{A simple font program table containing definitions (FDEFs) for functions 1 and 2.}
\end{figure}

Specifically, Figure 1 shows that the contents (data) of
the PUSH instructions outside of FDEF-ENDF pairs suffice to obtain all the
function labels for a font file. Since the font program table only
includes function definitions, the only instructions to pop
elements from the stack will be FDEF instructions, and their operands
will be function labels.

To compute the set of functions called, we need to check the bytecode from every
other table, including the bytecode of each glyph. Unlike function
definitions, however, we cannot easily obtain which labels are referred to by each
function call. That is because CALL instructions need not directly follow
PUSH instructions. Callees can be pushed on the stack anywhere in the
bytecode. Most often, they are pushed at the beginning of the
execution. However, intervening instructions will of course change the state of the stack
Moreover, intervening function calls can also change the current state of the stack---unlike
in many other stack-based languages, there is no requirement to preserve stack height across
function calls. Hence, when execution reaches a CALL instruction, the current value at the
top of the stack is not obvious.

\begin{figure}[ht!]
\begin{verbatim}
PUSH[ ]  /* 3 values pushed */
85 97 93
CALL[ ]
PUSH[ ]
85
CALL[ ]
CALL[ ]
\end{verbatim}
\caption{TrueType bytecode which calls three functions: 93, 85 and ?.
Unlike in other languages, the destination of the third call cannot be resolved
without analyzing the first two callees.}
\end{figure}

Figure 2 shows a program with three function calls. The program
initially pushes 3 values to the stack and makes the first function call.
The first function call will certainly resolve to
function 93, the value at the top of the stack
when the first CALL instruction is executed. The same goes for the
second CALL instruction, as the value 85 is pushed immediately before
it. However, for the third CALL instruction, we can not be sure about which
value is at the top of the stack because the second function call may have
changed the state of the stack.  

Therefore, to identify function calls within the bytecode of other tables,
we need to keep track of how many values are pushed and popped by each
instruction and by each function called. Then, whenever the bytecode invokes a CALL instruction,
we can look up the value at the top of the stack. This requires the use of our
Abstract Interpreter~\cite{bytecode}. Abstract
execution provides information about the stack height and approximate contents
after each instruction, as well as every possible branch that can be
executed, allowing us to estimate a list of function calls for each table. 

\section{Background}

A TrueType font file describes an outline font using a set of data tables.
Apple's TrueType Reference
Manual~\cite{ttmanual} describes all permissible tables.
In this work, we focus on the bytecode-containing fpgm, prep and glyf tables.
We are trying to reduce the fpgm table, which contains function definitions. The prep
table contains a set of instructions that are executed whenever the font
is first accessed, or when the font, point size, or transformation matrix
change. The glyf table describes glyphs in the
TrueType outline format.

To roughly estimate the contribution of each table to overall font
sizes, we used the Google Noto Sans~\cite{notosans} TrueType fonts as
a benchmark. For all inspected fonts, glyph-related tables (e.g. glyf, GPOS, GSUB, etc)
take together more than 50 percent of the total font size,
while the fpgm table never exceeds 10 percent.  

To perform the tree shaking on the font program table,
we use
Wenzhu Man's bytecode manipulation library~\cite{bytecode}
which builds on the open-source fonttools project~\cite{fonttools}.

Fonttools provides a TTFont class, which gives us easy access to the TrueType font format. It
allows us to extract data from the font tables, perform analysis on
them, and manipulate and save them back to the font. 

To get information about rendered glyphs and make sure that
our modifications do not cause any unexpected changes in the font file,
we use the FreeType interpreter~\cite{freetype} to load each glyph from the original
and reduced fonts at a given resolution and make sure they are exactly
the same. 

The Abstract Interpreter can be used to get information about the
program's execution. During abstract execution, instructions may return either
actual values (if available) or abstract data (otherwise). Although this kind of
execution provides less specific information about a specific execution than normal (concrete)
interpretation, we have found that it gives enough information to estimate font programs' control flow, including
which kind of instructions and functions were called.

The Abstract
Interpreter does not run using a TTFont object itself, but instead Wenzhu Man's
intermediate bytecodeContainer class.
The bytecodeContainer class makes font contents
more easily manipulable and allows translation back to TTFont
objects. The Abstract Interpreter uses the bytecodeContainer to execute font bytecodes.

\section{Removing unneeded functions}

The tree shaker starts by loading the font file as a TTFont object.
It then creates a bytecodeContainer object that encapsulates some of
the tables of a TrueType font file, including the prep, fpgm, cvt and
glyf tables. This object is necessary to run the abstract executor. We
also initialize the set of called functions with the empty set. This
set will determine which functions we keep in the font file.

Whenever a font is loaded, prep is always the first table to run to 
set up the environment for glyphs to be loaded. Note that, whenever we run
a table through the Abstract Interpreter, the resulting context
and information about the execution is stored in the interpreter object.
We are interested here specifically in function calls made during the execution.

\begin{figure}[ht!]
\centering
\includegraphics[height=100mm]{diagram.jpg}
\caption{Tree shake diagram \label{overflow}}
\end{figure}

The first step then is running the prep table through the Abstract 
Interpreter. We make a copy of the post-prep environment (we want to be
able to restore this initial state) and update our called function
set with the functions called by prep. We now have everything ready to
execute the remaining tables.

The second step is running every glyph through the Abstract Interpreter.
For each glyph we restore the execution context post-prep, execute the
glyphs, and update the called function set with calls made during
execution. Afterwards, we have all functions that can possibly be called
by that font.

The third and last step is obtaining the labels of the functions
to be removed (by subtracting the functions called from the list of
functions defined in the font's fpgm table). With them, we update
the bytecodeContainer object---removing the functions passed---and
translate the contents of bytecodeContainer back to the TTFont object. 
With that done, the TTFont object is ready to generate a reduced
tree-shaken font.

\section{Testing}

To verify that the tree shaking does not affect the rendering of any glyphs,
we developed a script to compare glyphs
from two fonts and check if they are identical. While we are using this
script to verify that nothing untoward happened during the tree shaking, it
applies to any lossless modifications applied to the glyphs of
a TrueType Font.

To perform the comparison, we used the FreeType interpreter to render
glyphs from both the original and reduced fonts for each charmap present
in the font. Detailed information about how the FreeType interpreter
renders glyphs can be found in the FreeType Reference Manual~\cite{freetypemanual}. 

The script compares the two fonts for a given set of glyphs at
five popular dpi resolutions: 72x72, 300x300, 600x600, 1200x1200 and
2400x2400dpi. Depending on the number of glyphs to be fetched, and from
where they have to be fetched from (file or URL), the test may take a
while. The default encoding is set to UTF-8. Both resolution and
enconding can also be specified via parameter.

\section{Results}

To measure the results of our work, we performed the tree shaking
technique on the Google Noto Sans fonts~\cite{notosans}. To obtain
better insights on how much space the font program table takes in the total
size of the font, we also applied the tree shaking on subsetted fonts
(fetching glyphs from the initial Wikipedia page from the corresponding
languages), to check against the reduction we obtained from the original
fonts.

\verbatiminput{averageStats}

The statistics show a slight variation between the reduction we obtained
with the tree shaking technique for whole fonts and for subsetted fonts.

Although we obtained a better relative reduction for subsetted fonts, the
reduction still averaged under 10 percent. For some fonts, like
NotoSansSinhala-Regular.ttf, the difference in relative reduction
between whole and subsetted font was less than 1 percent, and for fonts
like NotoSansGeorgian-Regular.ttf, this difference jumped to more than
10 percent. Appendix A contains detailed results for every font.

These results give us some insights on how much the function program
table affects the total size of the font, and more importantly, how
other tables contribute to the final size. We could observe that the
reduction we obtained from just subsetting the fonts greatly
varied, presenting reductions between 8 and 99 percent. The tree shaking
yields a better relative reduction in fonts that had their sizes
considerably reduced after subsetting, since the tables related to
glyphs usually take the major part of whole fonts.

Appendix B contains detailed information about how much each table takes
in the Google Noto Sans font files before and after subsetting.

\section{Conclusions and Future Work}

Tree shaking of font program tables of TrueType fonts helps to keep
them minimal, which may be useful for further manipulations such as
font merging. However, it proved to be not very efficient for size
reduction.

During our work, we had the chance to investigate the contents
of TrueType fonts, including how much each table takes in the final size
of the font. Our script can obtain such information from fonts in the
wild, which can be useful for further investigation of TrueType fonts.

Possible future works include improvements on the tree shake such as
identifying and removing duplicated functions from the font program
table, as well as the inspection of the other tables to identify
redundant bytecode or the search for new techniques to reduce these
tables.

\clearpage
\begin{appendices}
\section{Tree Shaking Statistics}

\verbatiminput{generalStats}

\section{Table Statistics}

To understand better how much the subsetting technique would affect
fonts in the wild, we inspected the size of each table before and after
subsetting.

To get information about the size of the tables, we used the TTX tool
from the fonttools project to dump the font file into an XML file. With
an XML file, we were able to count the number of bytes in each table.

Note that the information we get about sizes are not exact. The size
between the XML file and the original font file may differ (by a constant factor). Moreover, we
are interested in understanding the relative size of a table when
compared to the other tables in the font. So, even though we are not
getting precise values from the XML, it is still enough for
understanding how tables compare.

To make sure the glyphs would be rendered the same way as the original
fonts, we used the compare script to check for equality between original
and subsetted for the set of fetched glyphs. 

\verbatiminput{subsetStats}

\end{appendices}

\clearpage
\begin{thebibliography}{1}

\bibitem{ttmanual} Apple Inc. TrueType Reference Manual. 
\url{https://developer.apple.com/fonts/TrueType-Reference-Manual/} 

\bibitem{bytecode} Wenzhu Man. TrueType bytecode manipulation library. 
\url{https://github.com/wenzhuman/fonttools} 

\bibitem{freetype} The FreeType project. \url{http://www.freetype.org}

\bibitem{fonttools} The fonttools project. \url{https://github.com/behdad/fonttools} 

\bibitem{freetypemanual} FreeType Documentation.
\url{http://www.freetype.org/freetype2/docs/documentation.html}

\bibitem{notosans} Google Noto Sans. 
\url{https://www.google.com/fonts/specimen/Noto+Sans} 

\end{thebibliography}

\end{document}
